\documentclass[11pt]{article}
\usepackage{xcolor}
\usepackage[margin=1cm]{geometry}
\usepackage{array}

\definecolor{light-gray}{gray}{0.95}

\title{Algèbre linéaire}
\author{Axel Marmet }
\date{January 2019}

\begin{document}
\section*{\centering Axel Marmet}
\begin{tabular}{ >{\raggedright}m{0.48\textwidth} >{\raggedleft}m{0.48\textwidth} }
    10 Chemin de la Cleison \\ 1278 La Rippe & axel.marmet@epfl.ch \\ + 41 79 858 77 59\\
\end{tabular}
\begin{center}
    \rule{1\textwidth}{1pt}
\end{center}
\begin{minipage}[t]{0.33\textwidth}
    \subsection*{Semester 3}
    \begin{tabular}{ >{\raggedright}m{4cm} m{2cm} }
        Algorithms & $5.5$  \\ 
        Analysis III& $5.5$  \\  
        Computer architecture & $5.25$  \\  
        Computer networks & $5.75$  \\
        Functional programming & $5.75$ \\
        General physics : electromagnetism & $5.75$ \\
        Real-time systems & $5.75$  \\
        Sciences and religions A & $5.5$   
      \end{tabular}
      \\\\\\
      \subsection*{Semester 2}
      \begin{tabular}{ >{\raggedright}m{4cm} m{2cm} }
        Advanced information, \\computation, communication II & $5.75$  \\ 
        Analysis II & $5.25$  \\  
        Digital systems design & $5.5$  \\  
        Global issues : health A & $5.75$  \\
        Practice of object-oriented programming & $6$ \\
      \end{tabular}
      \\\\\\
      \subsection*{Semester 1}
      \begin{tabular}{ >{\raggedright}m{4cm} m{2cm} }
        Advanced information, \\computation, communication I & $4.75$  \\ 
        Analysis I & $5.25$  \\  
        General physics I & $3.5$  \\  
        Introduction to programming & $6$  \\
        Liner Algebra & $4.75$ \\
      \end{tabular}
\end{minipage}%
\begin{minipage}[t]{0.66\textwidth}
    \section*{Interests and Experience}
    What interests me the most in Computer Science and Mathematics is to create complex systems by composing simple elements together, 
    raising abstraction until . As of now I have been mainly able to do so in software, particularly in functionnal programming, be it Lisp or Scala.
    This way of creating small functionnal units and articulating them together using higher-order functions seems very intuitive and beautiful to me \\
    This interest of complex systems and languages then lead me to learn about metaprogramming. I have thus created a few compilers and interpreters in my free time.
    My favorite being a compiler taking in a simple language with flow control, arithmetic, variables and function calls and returning a gameboy ROM that would,
     when inserted in a virtual or physical gameboy, run the program.\\
    I think I am also drawn to theoretical computer science for the same reasons as one can prove theorems using axioms and definitions and then in turn prove more
    powerful results using previously proved theorems.

    \section*{Motivation}
    I am highly motivated for this internship for a few reasons. \\
    Firstly as stated in the previous section I have been able to enjoy creating complex systems in the software and theoretical computer science domains 
    but not yet as much as I would have wanted in hardware. As ........\\
    Moreover after discovering the subject of High-Level Synthesis I read a few introduction papers$^1$ and it
\end{minipage}
\\\\
\begin{center}
    \rule{\textwidth}{1pt}
\end{center}
$^1$Damaj, Issam. (2008). High‐Level Synthesis. 10.1002/9780470050118.ecse177. \\
$^1$Coussy, Philippe \& Gajski, Daniel \& Meredith, Michael \& Takach, Andres. (2009). An Introduction to High-Level Synthesis. Design \& Test of Computers, IEEE. 26. 8 - 17. 10.1109/MDT.2009.69. 
\end{document}
